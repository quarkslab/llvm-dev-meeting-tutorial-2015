\documentclass[14pt]{beamer}

\usepackage[english]{babel}
\usepackage[utf8x]{inputenc}

\usetheme{m}
\usepackage{pacman}
\usefonttheme{structurebold}
\setbeamercovered{transparent}


\usepackage{listings}
\definecolor{darkgreen}{rgb}{0,0.6,0}
\lstset{
    basicstyle=\ttfamily,
    showstringspaces=false,
    commentstyle=\color{red},
    keywordstyle=\color{blue},
    stringstyle=\color{darkgreen}
}

%%
% from http://tex.stackexchange.com/questions/23647/drawing-a-directory-listing-a-la-the-tree-command-in-tikz
\usepackage{forest}
\usetikzlibrary{arrows.meta}
\forestset{
  dir tree/.style={
    for tree={
      parent anchor=south west,
      child anchor=west,
      anchor=mid west,
      inner ysep=1pt,
      grow'=0,
      align=left,
      edge path={
        \noexpand\path [draw, \forestoption{edge}] (!u.parent anchor) ++(1em,0) |- (.child anchor)\forestoption{edge label};
      },
      font=\sffamily,
      if n children=0{}{
        delay={
          prepend={[,phantom, calign with current]}
        }
      },
      fit=band,
      before computing xy={
        l=2em
      }
    },
  }
}

%%

\newcommand{\Command}[1]{\textbf{\texttt{#1}}}
\newcommand{\Code}[1]{\textbf{\texttt{#1}}}

%%

\title{Building, Testing and Debugging a Simple out-of-tree LLVM Pass}
\date{October 29, 2015, LLVM Developers' Meeting}

\begin{document}

{
    \logo{\includegraphics[height=5em]{logo.png}\hspace{2em}}
	\begin{frame}
        \maketitle
	\end{frame}
}

    %%%%%%%%%%%%%%%%%%%%%%%%%%%%%%%%%%%%%%%
    %%%%%%%%%%%%%%%%%%%%%%%%%%%%%%%%%%%%%%%
    %%%%%%%%%%%%%%%%%%%%%%%%%%%%%%%%%%%%%%%

    \begin{frame}{LLVM 3.7 --- Tutorial}
        \begin{center}
            \textbf{\Large Press Start Button}
        \end{center}
    \end{frame}

    \begin{frame}{LLVM 3.7}
        \begin{center}
            \begin{itemize}
                \centering
                \item[]\alert{\bf Select difficulty}\vspace{1em}
                \item[] \textbf{\texttt{>~}Easy\texttt{~<}}
                \item[] Hard
                \item[] Nightmare
            \end{itemize}
        \end{center}
    \end{frame}

    \begin{frame}{LLVM 3.7}

            \begin{itemize}
                \centering
                \item[]\alert{\bf Stage Selection}\vspace{1em}
                \item[] Adding a new Front-End
                \item[] In-Tree Pass Developpment
                \item[] \textbf{\texttt{>~}Out-of-Tree Pass Developpment\texttt{~<}}
                \item[] Adding a new Back-End
            \end{itemize}

    \end{frame}

    \begin{frame}{LLVM 3.7}

            \begin{itemize}
                \centering
                \item[]\alert{\bf OS Selection}\vspace{1em}
                \item[] \textbf{\texttt{>~}Linux\texttt{~<}}
                \item[] OSX
                \item[] \tikz\node[opacity=0.5]{Windows};
            \end{itemize}

    \end{frame}

    %%%%%%%%%%%%%%%%%%%%%%%%%%%%%%%%%%%%%%%
    %%%%%%%%%%%%%%%%%%%%%%%%%%%%%%%%%%%%%%%
    %%%%%%%%%%%%%%%%%%%%%%%%%%%%%%%%%%%%%%%

    \begin{frame}{stage 1}

        \framesubtitle{Setup a Proper CMake Project}

        \begin{block}{Goals}
            \begin{itemize}
                \item Use LLVM cmake support
                \item Build a minimal pass
            \end{itemize}

        \end{block}

        \begin{alertblock}{Bonus}
            \begin{itemize}
                \item Setup a minimal test driver
                \item Make the pass compatible with \Command{clang}
            \end{itemize}
        \end{alertblock}

    \end{frame}

    \begin{frame}{stage 1 --- Directory Layout}
    \begin{forest}
          dir tree
          [Tutorial
              [CMakeLists.txt\only<2>{\usebeamercolor[fg]{alerted text}~$\longleftarrow$ CMake configuration file}]
              [cmake\only<3>{\usebeamercolor[fg]{alerted text}~$\longleftarrow$ CMake auxillary files}
                [Python.cmake]
              ]
              [MBA\only<4>{\usebeamercolor[fg]{alerted text}~$\longleftarrow$ Our first pass}
                  [CMakeLists.txt]
                  [MBA.cpp]
              ]
          ]
    \end{forest}
    \end{frame}

    \begin{frame}[containsverbatim]
        \frametitle{stage 1 --- \texttt{CMakeLists.txt}}
        \begin{block}{LLVM Detection}
            \footnotesize
            \lstinputlisting[firstline=5, lastline=11,language=bash,morekeywords={endif,message}]{../CMakeLists.txt}
        \end{block}
    \end{frame}

    \begin{frame}[containsverbatim]
        \frametitle{stage 1 --- \texttt{CMakeLists.txt}}
        \begin{block}{Load LLVM Config}
            \footnotesize
            \lstinputlisting[firstline=15, lastline=17,language=bash,morekeywords={list,find_package}]{../CMakeLists.txt}
        \end{block}
        \begin{block}{And more LLVM Stuff}
            \footnotesize
            \lstinputlisting[firstline=19, lastline=21,language=bash,morekeywords={list,include}]{../CMakeLists.txt}
        \end{block}
    \end{frame}

    \begin{frame}[containsverbatim]
        \frametitle{stage 1 --- \texttt{CMakeLists.txt}}
        \begin{block}{Propagate LLVM setup to our project}
            \footnotesize
            \lstinputlisting[firstline=24, lastline=27,language=bash,morekeywords={add_definitions,include_directories}]{../CMakeLists.txt}
        \end{block}

        \begin{block}{Get Ready!}
            \footnotesize
            \lstinputlisting[firstline=34, lastline=34,language=bash,morekeywords={add_subdirectory}]{../CMakeLists.txt}
        \end{block}
    \end{frame}

    \begin{frame}[containsverbatim]
        \frametitle{stage 1 --- \texttt{MBA/CMakeLists.txt}}
        \begin{block}{Declare a Pass}
            \footnotesize
            \lstinputlisting[firstline=2, lastline=2,language=bash,morekeywords={add_llvm_loadable_module}]{../MBA/CMakeLists.txt}
        \end{block}

        \begin{alertblock}{1 Pass = 1 Dynamically Loaded Library}
            \begin{itemize}
                \item Passes are loaded by a pass driver: \textbf{\texttt{opt}}
{
\footnotesize
\begin{lstlisting}[language=bash]
% opt -load LLVMMBA.so -mba input.ll -S
\end{lstlisting}
}
                \item Or by clang (provided an extra setup)
{
\footnotesize
\begin{lstlisting}[language=bash]
% clang -Xclang -load -Xclang LLVMMBA.so input.c -c
\end{lstlisting}
}
            \end{itemize}
        \end{alertblock}

    \end{frame}

    \begin{frame}[containsverbatim]
        \frametitle{stage 1 --- \texttt{MBA.cpp}}
        {
            \footnotesize
            \lstinputlisting[linerange={24-24,27-27,45-45,52-55,58-58,60-60,71-72,111-113},language=c++]{../MBA/MBA.cpp}
        }
    \end{frame}


    \begin{frame}[containsverbatim]
        \frametitle{stage 1 --- \texttt{MBA.cpp}}
        \begin{alertblock}{Registration Stuff}
            \begin{itemize}
                \item Only performs registration for \Command{opt} use!
                \item Uses a static constructor\dots
            \end{itemize}
        \end{alertblock}
        {
            \footnotesize
            \lstinputlisting[linerange={121-126},language=c++]{../MBA/MBA.cpp}
        }
    \end{frame}

    \begin{frame}[containsverbatim]
        \frametitle{stage 1 --- Bonus Level}
        \begin{alertblock}{Setup test infrastructure}
            \begin{itemize}
                \item Rely on \Command{lit}, LLVM's Integrated Tester
                \item {\footnotesize
\begin{lstlisting}[language=bash]
% pip install --user lit
\end{lstlisting}
                    }
            \end{itemize}
        \end{alertblock}

        \begin{block}{\texttt{CMakeLists.txt} update}
{
\scriptsize
\lstinputlisting[linerange={39-41,50-53},language=bash,morekeywords={list,include,find_python_module,REQUIRED,add_custom_target,COMMAND}]{../CMakeLists.txt}
}
        \end{block}
    \end{frame}

    \begin{frame}[containsverbatim]
        \frametitle{stage 1 --- Bonus Level}
        \begin{alertblock}{Make the pass usable from \Command{clang}}
            \begin{itemize}
                \item Automatically loaded in \Command{clang}'s optimization flow: {\footnotesize\lstinline|clang -Xclang -load -Xclang|}
                \item Several extension point exist
            \end{itemize}
        \end{alertblock}

{
\scriptsize
\lstinputlisting[linerange={130-136,139-140},language=bash,morekeywords={list,include,find_python_module,REQUIRED,add_custom_target,COMMAND}]{../MBA/MBA.cpp}
}
    \end{frame}

    %%%%%%%%%%%%%%%%%%%%%%%%%%%%%%%%%%%%%%%
    %%%%%%%%%%%%%%%%%%%%%%%%%%%%%%%%%%%%%%%
    %%%%%%%%%%%%%%%%%%%%%%%%%%%%%%%%%%%%%%%

    \begin{frame}{stage 2}

        \framesubtitle{Build a Simple Pass}

        \begin{block}{Goals}
            \begin{itemize}
                \item Learn basic LLVM IR manipulations
                \item Write a simple test
            \end{itemize}

        \end{block}

        \begin{alertblock}{Bonus}
            \begin{itemize}
                \item Follow LLVM's guidelines
                \item Collect statistics on your pass
                \item Collect debug informations on your pass
            \end{itemize}
        \end{alertblock}

    \end{frame}

    %%%%%%%%%%%%%%%%%%%%%%%%%%%%%%%%%%%%%%%
    %%%%%%%%%%%%%%%%%%%%%%%%%%%%%%%%%%%%%%%
    %%%%%%%%%%%%%%%%%%%%%%%%%%%%%%%%%%%%%%%

    \begin{frame}{stage 3}

        \framesubtitle{Build an Analysis}

        \begin{block}{Goals}
            \begin{itemize}
                \item Understand the difference with a Transformation
                \item Use Dominator trees
                \item Write a \Code{llvm::FunctionPass}
                \item Describe the dependency tree of a Pass
            \end{itemize}
        \end{block}

        \begin{alertblock}{Bonus}
            \begin{itemize}
                \item Follow LLVM's guidelines
            \end{itemize}
        \end{alertblock}


    \end{frame}

    %%%%%%%%%%%%%%%%%%%%%%%%%%%%%%%%%%%%%%%
    %%%%%%%%%%%%%%%%%%%%%%%%%%%%%%%%%%%%%%%
    %%%%%%%%%%%%%%%%%%%%%%%%%%%%%%%%%%%%%%%

    \begin{frame}{stage 3}

        \framesubtitle{Write a Complex Pass}

        \begin{block}{Goals}
            \begin{itemize}
                \item Modify the Control Flow Graph (CFG)
                \item Syndicate code in a third-party library
            \end{itemize}

        \end{block}

        \begin{alertblock}{Bonus}
            \begin{itemize}
                \item Declare extra options
            \end{itemize}
        \end{alertblock}

    \end{frame}

    %%%%%%%%%%%%%%%%%%%%%%%%%%%%%%%%%%%%%%%
    %%%%%%%%%%%%%%%%%%%%%%%%%%%%%%%%%%%%%%%
    %%%%%%%%%%%%%%%%%%%%%%%%%%%%%%%%%%%%%%%

    \begin{frame}{GAME OVER}

        \begin{itemize}
            \centering
            \item[]\includegraphics[height=2em]{logo.png}
            \item[]\alert{\bf Creditz}\vspace{-1em}
            \item[]\texttt{Serge~Guelton~<sguelton@quarkslab.com>}
            \item[]\texttt{Adrien Guinet <aguinet@quarkslab.com>}
        \end{itemize}
        %
        \begin{itemize}
            \centering
            \item[]\alert{\bf Insert Coins}\vspace{.1em}
            \item[]\texttt{Exit}
            \item[]\texttt{> Play Again <}
        \end{itemize}

    \end{frame}

\end{document}
